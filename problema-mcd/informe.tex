\documentclass[12pt]{article}
\usepackage[utf8]{inputenc}
\usepackage{manfnt}
\usepackage{amsfonts,amsmath, amsthm}
\usepackage[margin=2cm]{geometry}
\usepackage[spanish]{babel}
\usepackage{graphicx}

\begin{document}

  \title{Informe\\
  Matem\'atica Discreta I\\
      D. Same GCD's\\
     Problema:1295D}
  \author{Abel Molina S\'anchez\\
  Grupo 2-11\\
  Ciencias de la Computaci\'on\\
  Universidad de La Habana}
    \maketitle  

\newpage

\section{Problema}


\begin{center}

D. Same GCD's
\end{center}

You are given two integers $a$ and $m$. Calculate the number of integers $x$ such that $0\leq x<m$ and $gcd(a,m)=gcd(a+x,m)$.\\
Note: gcd(a,b) is the greatest common divisor of a and b.\\
\\
Input:\\
The first line contains the single integer $T$ ($1\leq T\leq 50$) — the number of test cases.\\
Next $T$ lines contain test cases — one per line.\\ Each line contains two integers $a$ and $m$ ($1\leq a<m\leq 10^{10}$).\\
\\
Output:\\
Print $T$ integers — one per test case. For each test case print the number of appropriate x-s.\\





\section{Insterpretación e idea}

El problema plantea de forma directa que se necesita determinar la cantidad de valores $x$, $0\leq x< m$ que cumplen que 
$mcd(a,m)=mcd(a+x,m)$.\\
Para determinar esta soluci\'on la primera idea de soluci\'on pasa por una iteraci\'on de fuerza bruta por todos los valores de $x$ en cada
 caso y para cada uno determinar si se cumple la igualdad de los mcd's utilizando el algoritmo de Euclides. Esta soluci\'on intuitiva no pasa 
por ser muy eficiente, con lo cual, la siguiente idea pasa por demostrar que la cantida de valores $x$ que cumplen la condici\'on es igual a
$\phi(m')$, donde $m'= m/mcd(a,m)$. Luego hay entonces que calcular $\phi(m')$ y para esto utilizaremos su f\'ormula en base a la descomposici\'on en factores primos de $m'$.\\  


\textit{\textbf{Lema1:} Sean $a,b\in N_+$ $a = da'$, $b= db'$. $d = mcd(a,b)\Rightarrow mcd(a',b')=1$.}\\
\textit{Demostraci\'on:} Tenemos $a = da'$, $b= db'$, sea $d = mcd(a,b)$ y $k=mcd(a',b')$, supongamos que $k\neq 1$. Luego tenemos que 
$a = dka"$ y $b = dkb"$, con lo cual $dk$ es divisor com\'un de $a$ y $b$, y $dk>d$ lo cual contradice que $d$ sea el $mcd(a,b)$.

\textit{\textbf{Lema2:} Sean $a,b,m\in N_+$ $mcd(ma,mb) = m*mcd(a,b)$}.\\
\textit{Demostraci\'on:} Sea $d = mcd(a,b)$, tenemos que:\\
 $ax + by = d$ (para $x,y\in Z$),\\
 $max + mby = md$\\
 $x(ma) + y(mb) = md$\\
 $mcd(ma,mb) = m*mcd(a,b)$\\
\\

\textit{\textbf{Lema3:} Sean $a,b,\in N_+$, $a|geq b$ $mcd(a,b) = mcd(a-b,b)$}.\\
\textit{Demostraci\'on:}
Sea $d$ un divisor común $a$ y $b$.\\
Luego se tiene: $a = da'$, $b = db'$ y por lo tanto
$a - b = da' - db' = d(a' - b') d$\\
Esto implica que $d$ es un divisor común de $(a - b)$ y $b$.\\
Ahora en sentido contrario tengamos que $d$ es un divisor común de $(a - b)$ y $b$.\\
 Entonces:\\
 $a - b = dp$ , $b = dq$, con lo cual\\
$a = (a - b) + b = dp + dq = d(p + q) $,\\
con lo cual $d$ es un divisor común de $a$ y $b$.\\
Así, los pares $(a, b)$ y $(a - b, b)$ tienen los mismos divisores comunes, en particular el $mcd$.\\
\\

\textit{\textbf{Lema4:} Sea $a\in N_+$, $\{r_1,r_2,....,r_{n}\}$ un Sistema Residual Completo m\'odulo $n$, 
$\{r_1 +a, r_2 + a,...., r_{n}+a\}$ es tambi\'en un Sistema Residual Completo m\'odulo $n$.}\\
\textit{Demostraci\'on:} Supongamos que existen \'indices $1\leq i,j\leq n$ tales que:\\
$ r_i + a \equiv r_j + a (n)$\\
entonces, $n$ divide  a $(r_i + a) - (r_j+a) = r_i-r_j$, con lo cual $n$ divide a $r_i-r_j$ lo que implica que:\\
$ r_i\equiv r_j(n)$, lo cual es una contradicci\'on con el hecho de que $\{r_1,r_2,..,r_n\}$ es un Sistema Residual Completo m\'odulo $n$.\\
\\

 Se busca determinar la cantidad de valores $1\leq x< m$ que cumplen que $mcd(a,m)=mcd(a+x,m)$. Luego tenemos que buscar una forma de
 determinar dicha cantidad. Para ello analizaremos las condiciones de los valores. Tenemos que los $x_i$ forman el menor SRC(m), y tenemos que
 $a+x\leq m$ o $a+x>m$. Si $a+x>m$, tenemos que se cumple que $mcd(a+x,m)=mcd(a+x-m,m)$, luego uniendo estos resultados tenemos que dicha 
 cantidad ser\'a el n\'umero de los $x'= (x+a)mod(n)$, $0\leq x\leq m-1$, tal mcd(x',m)=mcd(a,m).\\
 Luego sea $d = mcd(a,m)$, tenemos que:\\
 $a = da'$, $m = dm'$, con $mcd(a',m') = 1$.\\
 Como se busca $mcd(a,m)=mcd(x',m)$ , entonces, los n\'umeros cumplir\'an que:\\
 $d = mcd(x',m)$, lo que implica que $x'= dx"$ y como hemos visto $m= dm'$, donde $mcd(x",m')=1$.\\
 Luego tenemos que: \\
 $m=dm'$, $x'=dx"$, y $1\leq x'<m$, de donde $x'<m$, con lo cual, como $d=d$, $0\leq x"<m'$.\\
 luego, la cantidad de n\'umeros ser\'ia la cantidad de los $x"$ menores $m'$ primos relativos con $m'$. Pero esto es pr\'acticamente por definici\'on el valor de $\phi(m')$, faltar\'ia demostrar que nunca se alcanzar\'a la igualdad para $m'$, y esto es que $m'>1$, y esto se ve
 en el hecho de que $1\leq a<m$, con lo cual $d=mcd(a,m)<m$ y por tanto $m/d = m'>1 $.\\
 Luego hemos determinado que la cantidad bucada es igual a $\phi(m')$.\\
 \\

\newpage

\section{Pseudocódigo}


\begin{verbatim}

solve(a,m):
    d = mcd(a,m)
    return phi(m/d)

phi(m):
    phi = m
    div = 2
    while div*div <= m:
        if m%div==0:
            while m%div==0:
                m/= div
            phi -= phi/div
        div++
    if m > 1:
        phi -= phi/m
    return phi
    
\end{verbatim}



 Tenemos que, sea $n\in N$, $\phi(n) = n  \prod\limits_{i=1}^{k} (1 - \frac{1}{p_i})$, 
 con lo cual necesitamos analizar que se detectan los factores primos de m en la ejecuci\'on.\\
 \textit{Sea $n\in N_+$,  n tiene a lo sumo un factor primo mayor que $\sqrt{n}$}.\\
 Supongamos que tiene al menos dos, sean $p$ y $q$ . Se tiene que $p > \sqrt{n}$ y $q > \sqrt{n}$, con lo cual
  $p*q> \sqrt{n} *\sqrt{n}$ , con lo cual $p*q > n$ , lo cual contradice que $p$ y $q$ sean factores primos de $n$.\\
\\
Luego, tenemos que los divisores de cualquier n\'umero $n$ compuesto son $1\leq d_i\leq n$. Quitando los divisores triviales, se cumple que ser\'an: 
$1<d_i<n$. 
Luego tenemos que ning\'un n\'umero compuesto cumplir\'a la condici\'on \textit{$m\%div==0$},
ya que antes habr\'ian sido analizados sus divisores, y los divisores de un n\'umero $a$ son divisores de $a*k$ $\forall k\in N$. Por tanto tenemos que se detectan los factores primos de $m$ durante la 
iteraci\'on del ciclo, excepto a lo sumo uno, esto es, que haya uno mayor que $\sqrt{n}$.\\

Veamos que en caso de poseerlo, este factor se incluye en la condici\'on $if$ final. Digamos que el n\'umero $n$ posee un factor mayor que 
$\sqrt{n}$, sea ahora $n = \prod\limits_{i=1}^{j} p_{i}^{\alpha_i}$ la descomposici\'on en primos de $n$. Asumamos que los $p_i$ est\'an 
ordenados de menor a mayor en la productoria(el orden de los factores no altera el producto). Luego tendremos que 
$p_j > \sqrt{n}$, 
tenemos que $\alpha_j = 1$ porque de lo contrario tendr\'iamos al menos $p_{j}^{2} > n$. Luego, sea 
$q = \prod\limits_{i=1}^{j-1} p_{i}^{\alpha_i}$ , tenemos que $q|n$, $q*p_j=n$, luego $n/q = p_j$, y como en el ciclo se analizan los 
$p_i$, $i\in [1,...,j-1]$, entonces tenemos que en la condici\'on $if$ final detectamos a $p_j>\sqrt{n}$.\\
Por tanto luego de terminada la ejecuci\'on del algoritmo y por la f\'ormula de $\phi(m)$, habremos determinado el valor de $\phi$ para la 
entrada.\\
En el caso peor cada factor primo aparecer\'ia en la descomposici\'on con exponente $\alpha_i=1$, con lo cual se completa un ciclo hasta la 
iteraci\'on $i = \sqrt{m}$, con lo cual la complejidad temporal de $phi(m)$ es $O(\sqrt{m})$.\\
Luego, para analizar la complejidad temporal de la soluci\'on de cada test, tenemos que la complejidad de determinar el mcd(a,m) por el 
algoritmo de Euclides ser\'a $O(log(m))$, y la complejidad temporal de \textit{phi} ser\'a en $O(\sqrt{m'})$, donde $m' = m/mcd(a,m)$, luego
 la complejidad de la soluci\'on de cada test ser\'a de $O(logm + \sqrt{m'})$ de donde terminar completa la ejecuci\'on ser\'a 
$O(T*(log m + \sqrt{m'}))$.\\

\section{Tester}

Para la realizaci\'on del tester, se utiliza la soluci\'on de fuerza bruta del problema para determinar la soluci\'on esperada para cada 
entrada. En la carpeta \textit{./test} dentro de \textit{./test-casos-prueba} se deben encontrar por cada caso de prueba 3 archivos: casoX.in,
casoXexpected.out y casoX.out; X identifica el n\'umero del caso, el .in contiene los valores de entrada del caso, ..expected.out debe contener
 la soluci\'on dada por el algoritmo de fuerza bruta(incluido dentro de la carpeta tester), y el casoX.out debe tener la soluci\'on dada por 
el algoritmo de soluci\'on.\\
Para generar las entradas de los casos se cre\'o el \textit{generator.py} en el cual se da la opci\'on de crear un caso de prueba de forma 
manual o generar uno aleatorio utilizando el m\'etodo \textit{randint} de la librer\'ia \textit{random} de \textit{python}.\\
Para comprobar la soluciones se debe ejecutar el script \textit{checker.py} el cual por cada caso de prueba determina que la soluci\'on 
ofrecida por el algoritmo coincide con la esperada y por lo tanto es soluci\'on. Hay que tener en cuenta que  \textit{./test} debe incluir los 3 archivos de cada caso, y que \textit{checker.py} analizar\'a todos los casos incluidos en \textit{./test}. En el caso de prueba n\'umero $9$ se utilizan como entrada del algoritmo los $30$
 primeros n\'umeros de fibonacci. Los casos del 2 al 8 son tomados del Codeforce.\\



  










\end{document}