\documentclass[12pt]{article}
\usepackage[utf8]{inputenc}
\usepackage{manfnt}
\usepackage{amsfonts,amsmath, amsthm}
\usepackage[margin=2cm]{geometry}
\usepackage[spanish]{babel}
\usepackage{graphicx}

\begin{document}

  \title{Informe\\
  Matem\'atica Discreta I\\
     Pashmak and Buses\\
     Problema:459C}
  \author{Abel Molina S\'anchez\\
  Grupo 2-11\\
  Ciencias de la Computaci\'on\\
  Universidad de La Habana}
    \maketitle  

\newpage

\section{Problema}

\begin{center}
                   
C. Pashmak and Buses

\end{center}

Recently Pashmak has been employed in a transportation company. The company has $k$ buses and has a contract with a school 
which has $n$ students. The school planned to take the students to $d$ different places for $d$ days (each day in one place). 
Each day the company provides all the buses for the trip. Pashmak has to arrange the students in the buses. He wants to arrange the students in a 
way that no two students become close friends. In his ridiculous idea, two students will become close friends if and only if they are in the same buses
 for all d days.\\
Please help Pashmak with his weird idea. Assume that each bus has an unlimited capacity.\\
\\
Input:\\
The first line of input contains three space-separated integers $n,k,d$
($1\leq n$, $d\leq 1000$;$1\leq k\leq 10^9$)\\
\\
Output\\
If there is no valid arrangement just print $-1$. 
Otherwise print $d$ lines, in each of them print $n$ integers. The \textit{j-th} integer of the 
\textit{i-th} line shows which bus the \textit{j-th} student has to take 
on the \textit{i-th} day. You can assume that the buses are numbered from $1$ to $k$.\\


\section{Interpretación e idea}
  
Se tienen $n$ alumnos, $d$ d\'ias y $k$ autobuses. Se necesita saber si existe al menos una forma o no de asignar los autobuses a cada estudiante durante los $d$ d\'ias de tal forma que no existan $2$ estudiantes que coincidan cada 
d\'ia en los mismos autobuses.\\
Esto es, se necesita conocer si existen al menos $n$ distribuciones distintas de los $k$ autobuses en los $d$ d\'ias.
 En caso de existir hay que generar una de ellas.\\
\textit{Idea:}\\
Conociendo la forma de calcular la cantidad de $d$-permutaciones de un conjunto de tama\~no $k$, por el principio del palomar se puede conocer si existe una posible soluci\'on al problema, ya que en caso de haber menos de $n$ formas distintas de distribuci\'on es seguro que al menos $2$ alumnos poseer\'an la misma distribuci\'on.\\
Luego para obtener una de las posibles soluciones se busca generar las primeras $n$ distribuciones de $d$ 
posiciones donde cada posici\'on va de $1$ a $k$.\\

\newpage

\section{Algoritmo y correctitud}

\begin{verbatim}

solve(n,k,d):
    if n > k**d:
        fin no solucion
    D = matriz[d][n]
    for i in [0,...,d-1]
        for j in [0,...,n-1]
            D[i,j] = 1
    distribution = [1 for i in [0,..,d-1]]
    for c in [1,....,n-1]
        for j in [d-1,d-2,...,0]
            distribution[j] = (distribution[j]%k)+1
                if distribution[j] != 1
                    break
        for j in [0,...,d-1]
            D[j][c] = distribution[j]
    return D 
\end{verbatim}


Llamaremos distribuci\'on a una fila de $d$ valores, donde cada posici\'on asume valores en $[1,k]$ y que constituye la repartici\'on de los autobuses por d\'ias de un alumno. O sea, en cada uno de los $d$ d\'ias hay un autob\'us asignado.\\
No hay restricciones en cuanto a capacidad de los autobuses, por lo cual todos los alumnos podr\'ian tomar para
 un d\'ia $d_i$ el mismo autob\'us.\\
\textit{Luego diremos que dos distribuciones, $d1$ y $d2$, son distintas si para una posici\'on $i$, $d1_i\neq d2_i$. O sea, si no tienen igual valor en todas las posiciones.}\\
\textit{Diremos que una distribuci\'on $d1$ es mayor que una distribuci\'on $d2$ si en el recorrido i=[0,..,d-1], en la primera posici\'on donde $d1_i\neq d2_i$, se tiene que $d1_i > d2_i$.}\\

Una vez vistas las definiciones , tenemos que buscar una distribuci\'on distinta para cada alumno. 
Tenemos que el n\'umero de posibles distribuciones ser\'a igual a $k^d$ que es el n\'umero de $d$-permutaciones con repetici\'on de un conjunto de tama\~no k (cada posici\'on $d_i$ tiene $k$ opciones de autobuses, $1$ 
entre $\{1,2,...,k\}$, con lo cual $k*k...*k = k^d$).\\
Luego habr\'a soluci\'on si logramos que exista una distribuci\'on distinta para cada alumno, luego como hay $n$ alumnos y $k^d$ distribuciones, por el Principio del palomar sabemos que para tener una soluci\'on 
es necesario que $n\leq k^d$ ya que si hay m\'as alumnos que posibles distribuciones, al menos 2 alumnos tendr\'an la misma distribuci\'on, con lo cual no habr\'ia soluci\'on.\\ 
Luego  en la condici\'on \textit{if} inicial se comprueba la existencia o no de soluci\'on.\\
Ahora necesitamos ver que las $n$ distribuciones que se crean en el ciclo son v\'alidas y distintas.\\
Que son v\'alidas se garantiza porque tenemos que los restos con $k$ son $[0,1,...,k-1]$, y por tanto los restos $+1$ estar\'an en el rango $[1,...,k]$ y todos se alcanzan porque sabemos que $i\in[1,...,k-1]$ deja 
resto $i$ con $k$ y $k\%k=0$.\\
Luego para demostrar que las distribuciones formadas son distintas nos apoyaremos en las comparaciones definidas anteriormente, demostrando que cada distribuci\'on es mayor que la anterior. Vamos a hacerlo por inducci\'on en la iteraci\'on del donde se forma la distribuci\'on.\\
 Tenemos que en un primer momento se tiene la distribuci\'on $d1 = [d_i=1$ $\forall i\in [0,..,d-1]]$ que es una posible. Luego la primera que se forma en el ciclo, es la distribuci\'on \\
 $ d2 = [d_0=1,d_1=1,...,d_{d-2}=1,d_{d-1} = 2]$, que es mayor que $d1$.\\ 
 Ahora asumamos por hip\'otesis que se cumple para la distribuci\'on $dk$.\\ 
 Sea ahora el momento donde se est\'a formando la distribuci\'on $dk+1$, tenemos que en ese momento \textit{distribution} es igual a $dk$. Luego, como se recorren las posiciones desde $[d-1,..,0]$, sea $j$ la primera posici\'on donde \textit{$distribution[j]\neq 1$}. Tenemos que $distribution_{i+1} [j]$ se hace $1$ solo cuando $distribution_{i} [j] = 5$, porque en $[1,...,k]$ que son los valores alcanzables solo $k\%k=0$ y por tanto 
 $k\%k+1=1$.\\
 Luego tenemos que para $j$ $dk[j]= i \in[1,...,k-1]$, por tanto $i\%k=i$ e $i + 1 > i$, con lo cual la 
 distrubuci\'on $dk+1$ es mayor que la distribuci\'on de $k$. Luego queda demostrado que cada distribuci\'on es mayor que la anterior y por tanto distintas.\\
 
   
\textit{Complejidad temporal}\\

 Determinar si se cumple o no el principio del palomar es $O(d)$.\\
 Luego para generar las $n$ distribuciones de los estudiantes se realizan dos ciclos $for$ anidados, el primero realiza $O(n)$ iteraciones y el segundo $O(d)$ iteraciones por lo cual completar las distribuciones es $O(n*d)$.\\
 Luego para imprimir la solucion se recorre todas las distribuciones, con lo cual se realiza nuevamente en $O(n*d)$, con lo cual por regla de la suma se tiene que la complejidad temporal de la soluci\'on es $O(n*d)$.\\
 \\

\section{Tester}


 Como parte de la realizaci\'on del ejercicio se incluye un tester y un generador de casos. Ambos estan incluidos en la carpeta \textit{tester-casos-prueba} junto con una copia de la soluci\'on.\\
 Dentro del generador se incluyen dos m\'etodos para generar una entrada para el problema. 
 Uno crea el archivo \textit{casoX.in} con los parametros elegidos de forma manual, y el otro genera la entreda generando 3 enteros de forma aleatoria utilizando el m\'etodo \textit{randint(a,b)} de la libreria random de python.\\
El \textit{tester.py} para comprobar las soluciones de cada caso de prueba:\\
Lee la entrada y comprueba por el Principio del polomar que el caso deba tener o no soluci\'on, en caso de no tenerla comprueba que la salida del algoritmo haya sido '-1'.\\
 Luego lee la salida completa del algoritmo y comprueba que haya sido con el formato correcto, esto es: 
 $d$ filas con $n$ columnas.\\
 Se crea un \textit{set} para almacenar las distribuciones de cada estudiante y comprobar que no hay 2 iguales. Se recorre la matriz de soluci\'on comprobando que cada n\'umero en cada distribuci\'on est\'a 
 entre $1$ y $k$(representan un autob\'us valido) y se van agregando las distribuciones al set, si se detectan 2 iguales en alg\'un momento se determina una soluci\'on err\'onea, en caso contrario luego de terminada la comprobaci\'on la soluci\'on ser\'ia correcta.\\
El tester comprobar\'a todos los tests incluidos en la carpeta \textit{./tests} con un '.out'. Por tanto es importante que si se agrega alg\'un nuevo caso para trabajar la soluci\'on este tenga un '.in' y un '.out' con igual nombre y diferente al del resto de casos dentro de la carpeta \textit{tester}.\\
 Como casos de pruebas que se consideran importantes analizar est\'an aquellos que puedan cosiderarse extremos dentro de la ejecuci\'on. 
 Comprobamos para cuando $n=k=d=1$(caso1), comprobamos el test de estr\'es del algoritmo para cuando cada par\'ametro recibe el m\'aximo valor posible dentro de las restricciones del problema, esto es $n=d=1000$ y $k=10^9$(caso3). 
 Se comprueba para cuando no existe soluci\'on y deja de existir por una diferencia m\'inima, esto es, cuando $n=k^d + 1$, lo hacemos con $n = 65$, $k=4$, $d=3$(caso4). Comprobamos cuando $n=k^d$, lo hacemos con $n=64(27)$, $k=4(3)$, $d=3(3)$(casos 5 y (3)).\\

\end{document}