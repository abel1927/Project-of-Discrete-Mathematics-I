\documentclass[12pt]{article}
\usepackage[utf8]{inputenc}
\usepackage{manfnt}
\usepackage{amsfonts,amsmath, amsthm,amssymb}
\usepackage[margin=2cm]{geometry}
\usepackage[spanish]{babel}
\usepackage{graphicx}

\begin{document}

  \title{Informe\\
  Matem\'atica Discreta I\\
     Two Divisors\\
     Problema:1366D}
  \author{Abel Molina S\'anchez\\
  Grupo 2-11\\
  Ciencias de la Computaci\'on\\
  Universidad de La Habana}
    \maketitle  

\newpage

\section{Problema}

\begin{center}

D.Two Divisors

\end{center}

You are given $n$ integers $a_1,a_2,…,a_n$.\\
For each $a_i$ find its two divisors $d1>1$ and $d2>1$ such that $gcd(d1+d2,ai)=1$ (where $gcd(a,b)$ is the greatest common divisor 
of $a$ and $b$) or say that there is no such pair.\\
\\
Input:\\
The first line contains single integer $n(1\leq n\leq 5*10^5)$ — the size of the array $a$.
The second line contains $n$ integers $a_1,a_2,…,a_n$ $(2\leq a_i\leq 10^7)$ — the array $a$.\\
\\
Output:\\
To speed up the output, print two lines with $n$ integers in each line.\\
The $i-th$ integers in the first and second lines should be corresponding divisors $d1>1$ and $d2>1$ such that 
 $gcd(d1+d2,ai)=1$ or $-1$ and $-1$ if there is no such pair. If there are multiple answers, print any of them.\\ 




\section{Interpretación e Idea}

El problema plantea el hecho de encontrar para cada n\'umero $a_i$ si posee al menos dos divisores $d1$, $d2$ que cumplen que 
$mcd(d1+d2, a_i)=1$.\\
La primera idea de soluci\'on y a modo de fuerza bruta ser\'ia determinar todos los divisores de $a_i$ y luego determinar dos a dos si cumplen
 que $mcd(d_i+d_j,a_i)=1$ utilizando el algoritmo de Euclides para el c\'alculo del $mcd$.\\
Luego en busca de mejorar la eficiencia de la soluci\'on vamos a analizar la descomposici\'on en factores primos de $a_i$, para determinar que 
para cuando $a_i=p_i^{\alpha_i}$ con $p_i$ primo, no existir\'an tales divisores; y en caso contrario vamos a demostrar que escogiendo un 
$d1=p_j$, $p_j$ factor primo de $a_i$, y $d_2=a_i/p_{j}^{\alpha_j}$, tendremos que $d1$ y $d2$ son divisores que cumplen el criterio.\\
Luego a la hora de la implementaci\'on apoyandonos en el algoritmo de la criba de Eratosthenes vamos a predeterminar para cada valor posible 
de la entrada su menor factor primo. Luego una vez tenido este, lo tomamos como $d1$ y de la forma descrita anteriormente obtenemos $d2$.\\
\\


\section{Demostración}

\textit{\textbf{Teorema1:} Sea $n = \prod\limits_{i=1}^{k} p_{i}^{\alpha_i}$ la descomposici\'on en factores primos de $n\in N_+$,
 si $k=1$, no existen dos divisores $d1$, $d2$ de $n$ que cumplen que $mcd(d1+d2,n)=1$.}\\
\textit{Demostraci\'on:} Como $k = 1$, $n = p_{1}^{\alpha_1}$, luego tendremos que todo dividsor $q$ de $n$ es divisible por $p_1$, porque, 
sea un $q'$, $q'| n$, si $q'$ es primo $q'=p_1$, luego no puede ser primo, entonces, $q'$ compuesto y sea, 
$q"$ el divisor no trivial m\'as peque\~no de $q'$, tenemos que $q"$ es primo, y $q"|n$ entonces $q"=p_1$.\\
Luego sean dos divisores cualesquiera de $n$, $d1$ y $d2$, $d1=p_1*d1'$, $d2=p_1*d2'$, luego,\\
 $d1+d2=p_1(d1'+d2')$ y por tanto 
$p_1| d1+d2$. Con lo cual queda demostrado.      


\textit{\textbf{Teorema2:} Sea $p_{1}^{\alpha_1}p_{_2}^{\alpha_2}....p_{_k}^{\alpha_k}$ la descomposici\'on en factores primos de $n\in N_+$,
 si $k>1$, existen dos divisores $d1$, $d2$ de $n$ que cumplen que $mcd(d1+d2,n)=1$.}\\
\textit{Demostraci\'on:} Sea $n = p_{1}^{\alpha_1}p_{_2}^{\alpha_2}....p_{_k}^{\alpha_k}$, $k>1$,\\
 $n = \prod\limits_{i=1}^{k} p_{i}^{\alpha_i} $, \\
sea $p_j$ arbitrario, uno de los factores primos de $n$, tenemos:\\
 $n = p_{j}^{\alpha_j}*\prod\limits_{i\neq j}^{k} p_{i}^{\alpha_i} $ \\
 sea $q_j =  \prod\limits_{i\neq j}^{k} p_{i}^{\alpha_i} $ \\
 $n = p_{j}^{\alpha_j}* q_j $ \\
 Sea ahora $p = p_i, i\in[1,..,k]$ arbitrario, uno de los factores primos de $n$.
 %tenemos que $p_j| p_{j}^{\alpha_j}$, $p_j | n$\\
 tenemos que $p_j \nmid  q_j$, porque sabemos que si $p_j| q_j \Rightarrow p_j| p_i$ para un $i\neq j \in [1,..,k]$, lo cual ser\'ia una 
contradicci\'on con el hecho de que los $p_i$ son primos.\\
 a su vez, $q_j \nmid p_j $, porque entonces $q_j = p_j$, no puede ser $1$ porque $k>1$, y por tanto los $p_i\in q_j$ dividen a $p_j$, lo cual es una contradicci\'on con que $p_j$ es primo.\\
 Por tanto tenemos que $p$ divide a uno entre $p_j$ y $q_j$, si $p|p_j$, $p = p_j$ y $p_j\nmid q_j$; y si 
 $p\nmid p_j$, $p=p_i\neq p_j$, por 
tanto $p| q_j$. Sin p\'erdida de generalidad digamos que $p|q_j$.\\
 Luego tenemos:\\
 $q_j\equiv 0(p)$\\
 $p_j\equiv r(p), r\neq 0$\\
 $p_j + q_j \equiv r + 0(p)$\\
 $p_j + q_j \equiv r(p), r\neq 0$\\
 Luego $p\nmid p_j + q_j$ [1]\\
 Luego tenemos que ning\'un $p_i$ divide a $p_j+q_j$.\\
 Luego supongamos que $mcd(p_j+q_j,n)= d > 1$. \\
 $d| n $ y $d| p_j + q_j$, \\
 Si $d$ es primo es una contradicci\'on con [1], por tanto analicemos para cuando $d$ es compuesto.\\
  sea ahora $p'$ el menor divisor no trivial de $d$, sabemos que $p'$ es primo.\\ 
 $p'|d $, luego $p'|n$ y $p' | p_j + q_j $, pero entonces $p'$ es un factor primo de $n$, con lo cual se contradice [1]. Luego, hemos encontrado dos divisores, $d1=p_j$ y $d2=q_j$ de $n$ que cumplen que $mcd(d1+d2,n)=1$, y a su vez hemos determinado una forma de hallarlos
teniendo en cuenta la elecci\'on arbitraria de $p_j$.\\

\newpage

\section{Algoritmo y complejidad temporal}


\begin{verbatim}

solve(n,a):                                      |sieve_e(N):
    N = 10**7+1                                  |    p = 2
    min_prime = [0]*N                            |    while p*p <= N:
    sieve_e(N)                                   |        if min_prime[p]:
    d1,d2 = [-1]*n,[-1]*n                        |            continue
    for i in [0,...,n-1]:                        |        for i = p*p; i < N; i+=p:
        p = min_prime[ai]                        |            if not min_prime[i]:        
        if ai is prime:                          |                min_prime[i] = p
            continue                             |        p++
        while p | ai:                            |
            ai = ai/p                            |
        if ai > 1                                |
            d1[i],d2[i] = p,ai                   |
     return d1,d2                                |
\end{verbatim}


Veamos que la criba nos coloca en $min-prime[i]$ el valor del menor factor primo de $i$ o $0$ si $i$ es primo. En un principio todos los 
n\'umeros en el rango hasta N se marcan como primos, ahora tenemos que, los divisores de un n\'umero compuesto $c$ son $\leq c$. En particular sus factores primos. A su vez tenemos que cualquir n\'umero compuesto $n$ tiene al menos un factor primo $\leq \sqrt{n}$.
Para demostrarlo supongamos lo contrario, esto es, tengamos el n\'umero $n$ con factorizaci\'on en primos $n=p_1p_2...p_r$.\\
Supongamos que $p_i>\sqrt{n}$ $\forall i=1,..,r$. Entonces tendr\'iamos que :\\
$p_1*p_2....*p_r > (\sqrt{n})^r = n^{r/2} \geq n$ si $r\geq 2$ y tenemos que $n$ es compuesto con lo cual $r\geq 2$ y llegamos a una 
contradicci\'on. \\
Luego se tiene que se alcanzan los primos $\leq {N}$ en el ciclo hasta $\sqrt{N}$ y para cada $n<N$ sabemos que
 $\sqrt{n}<\sqrt{N}$. Veamos entonces que para cada valor compuesto se actualiza este valor, y esto ocurre en el ciclo interno. Para eso necesitamos ver que luego de determinar los m\'ultiplos de los primos $2,3,..,P_k$, tenemos que los n\'umeros compuestos que quedan sin analizar tienen factores solamente mayores que $P_k$, de lo contrario habr\'ian sido analizados como m\'ultiplos de un $p_i\leq P_k$.  A su vez se tiene que el menor de estos compuestos es $(P_{k+1})^{2}$, ya que todos los m\'ultiplos de $P_{k+1}$ fueron analizados con los m\'ultiplos de los $p_i\leq P_k$. A su vez quedar\'an en $0$ los primos en el rango $P_k$, $P_{k+1}$ ya que no tienen divisores no triviales por lo cual no 
ser\'an m\'ultiplo de ning\'un $p_i$. Para finalizar es claro que hasta $\sqrt{N}$ se alcanzan todos los compuestos ya que a lo sumo $P_k = \sqrt{N}$ y se tendr\'ia que 
$(P_{k})^{2}=N$. Con lo cual luego de finalizado se habr\'an analizado todos los n\'umeros compuestos en el rango hasta $N$ y los primos quedar\'an en $0$.\\
Luego se elige $d1 = min-prime[i]$ para cada $a_i$ y garantizamos en el \'ultimo ciclo while que se tenga 
$d2 = ai/p^{\alpha}$ que ser\'a soluci\'on si $a_i \neq p^{\alpha}$ por lo demostrado en los teoremas 1 y 2.\\
       

Para el an\'alisis temporal de la soluci\'on tenemos que el algoritmo de la criba de Eratosthenes es 
$O(NloglogN)$\footnote{Time-complexity-of-Sieve-Eratosthenes-GeeksforGeeks.pdf en la bibliograf\'ia, tomado del sitio \textit{www.geeksforgeeks.org}}, donde tenemos que 
$N$ es el l\'imite superior de la entrada del algoritmo que ser\'a $a_i\leq 10^7$. Luego, tenemos que obtener $d1=p$ se hace en $O(1)$. En el segundo ciclo para la b\'usqueda de $d2$ tenemos el peor caso cuando $a_i$ es de la forma $p_{i}^{k}$ y tenemos que se realiza la mayor cantidad de iteraciones cuando es de la forma $2^k$. Luego tenemos que determinar la soluci\'on para los $n$ $a_i$ es $O(nlogN)$. Luego las inicializaciones son $O(N)$ y $O(n)$ al igual que devolver la soluci\'on que ser\'a $O(n)$. Luego 
tenemos que complejidad temporal de la soluci\'on ser\'a $O(NloglogN+nlogN)$.\\
\\


\section{Tester}

Se incluye un \textit{generador.py} para generar casos de forma aleatoria y de forma manual. Para analizar en los casos en la carpeta 
\textit{./tests} se debe encontrar el archivo \textit{casoX.in} y el archivo \textit{casoX.out}, este \'ultimo debe contener la respuesta 
dada por el algoritmo a la entrada en el .in.\\
Para analizar la correctitud de las soluciones se incluye el script \textit{checker.py} que analiza cada caso de prueba que se encuentra en 
\textit{./tests}.\\
Para determinar la correctitud de la soluci\'on, en primer lugar se leen las listas de entrada $a$, y las listas de soluci\'on $d1$ y $d2$ 
del algoritmo.\\
Por cada caso se itera por las listas, si se tiene que los divisores para $a_i$ son distintos de $-1$, se comprueba, empleando el m\'etodo
 $gdc(a,b)$ de la librer\'ia $math$ de $python$ si el $mcd(d1+d2,a_i)\neq 1$ y que $d1|a_i$ y $d2|a_i$. En caso de que la soluci\'on dada por el
 algoritmo sea que no existen $d1$ y $d2$ para $a_i$, entonces se utiliza la soluci\'on de fuerza bruta para determinar que esto es correcto.
 Para ello se tiene el m\'etodo \textit{brute-force} que crea una lista para almacenar los divisores no triviales de $a_i$ en cada caso.
 Para esto, se itera desde $div=2$ mientras $div\leq \sqrt{a_i}$ y si $div|a_i$ se agrega a lista de divisores, y si $div\neq \sqrt{a_i}$, se agrega $div2= a_i/div$ a la lista. Luego se comprueba que el $mcd(div_i + div_j, a_i)\neq 1$ para cualquier par de divisores $(div_i,div_j)$.\\
Se incluyen algunos casos de pruebas, como por ejemplo el $caso4$ donde se analiza el algoritmo para todos los 
n\'umeros en el rango $[1,5*10^5]$\footnote{Probablemente al momento de recibir el informe haya 2 o 3 casos solamente pues los casos grandes generan ficheros de mucho peso para el env\'io por correo. La idea inicial fue analizar todos los rangos de 500000 naturales desde 1 hasta $10^7$ y se cubrir\'ian todas las posibles entradas del problema.}


\end{document}